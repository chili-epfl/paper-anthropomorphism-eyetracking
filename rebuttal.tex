\documentclass{article}
\usepackage[utf8]{inputenc}


\begin{document}

Dear Editor, Dear Reviewers,

\vspace{2em}

We would like first to thank the editor and the reviewers for their thorough review of our
work. The manuscript underwent significant changes: we have hopefully addressed most of
the reviewers' concerns, and we discuss hereafter the remaining points.

The core points raised by the editors were the following:

\begin{itemize}
    \item \emph{Lack of theoretical background for the concept of "cognitive
        context"}: we take this point, and have reframed our theoretical
        background around "cognitive priming". Section 1.1 accordingly provides
        the literature references and precisely define our shallow vs deep
        conditions.
    \item \emph{Validity of the manipulation check} TBD
    \item \emph{Missing material}: We have appended the pre- and
        post-questionnaires to the manuscript appendices. We have as well added
        a link to the video stimuli, and eye-tracking and questionnaires
        results.
    \item \emph{Cursory eye-tracking results}: we have significantly expanded
        our reporting of eye-tracking results [TBD add details]
    \item \emph{Curiosity about cognitive skills vs actual ascription of cognitive
        skills}: one question raised concerns whether the participants are
        actually ascribing cognitive capacities to the robot, or merely curious
        whether the robot has them or not. While our approach does not
        intrinsically allow to rule out one or the other of these
        interpretations, we argue that this issue is to be observed in virtually
        any short-term human-robot interaction: as we previously
        argued~\cite{lemaignan2014cognitive}, the initial human-robot cognitive
        familiarisation period is basically about restating and iteratively
        shaping a mental model of the robot. During that period of time,
        anthropomorphic \emph{projections} are likely not yet anthropomorphic
        \emph{ascriptions}, and any investigation of anthropomorphism at this
        stage is subject to this double interpretation ``curiosity'' vs
        ``ascription''. In order to rule out one of the interpretation,
        longer-term observations (once the novelty effect has vanished) are
        needed, which could be relatively easily conducted with our method (by
        relying on non-intrusive gaze tracking techniques). [TBD: Kshitij
        argument]
    \item \emph{Use of the concept of proxy}: Using the term ``proxy'' was a
        mistake, rightfully pointed by the editor. This has been removed from
        the text.

\end{itemize}

The introduction of the article has been largely rewritten, to better frame our
work, and to clarify our working hypotheses.


We have as well attempted to address individually each of the other concerns
raised by the reviewers in this document (note that minor typos/style issues
have been fixed directly in the manuscript, and are not discussed here).


\section{Comments from Editor}


We list hereafter the main points raised by the editor (the other ones, omitted
here for brevity, have been directly addressed in the text):

\begin{itemize}
    \item \emph{Computation of the gazing differences between the human and the
        robot conditions}: TBD
    \item \emph{Choice of the questionnaire result subset}: TBD
    \item \emph{Possible ordering effect in the manipulation check}: TBD
    \item \emph{Interpretation of the head fixation in the 'deep cognitive
        priming' condition}: the editor suggests that the interpretation of the
        head focus in terms of higher anthropomorphic ascriptions onto the robot
        depends on the meaningfulness of such a head focus for a human. As noted
        in the paper, we did not observed significant differences in fixations
        on the human head between the shallow and deep cognitive conditions. We
        interpret this absence of difference as the result of the mere presence
        of a human being: because of the human presence, the two conditions
        (deep vs shallow) do not elicit anymore a significantly different perception of
        the situation as being machine-like or human-like. As such
        such a strong intrinsic source of 

\end{itemize}

All the other grammar and style errors pointed by the editor have been addressed.

\bibliographystyle{abbrv}
\bibliography{biblio}



\end{document}
