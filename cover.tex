\documentclass{article}

\usepackage[utf8]{inputenc}

\begin{document}

\section{Authors List}

\begin{itemize}
    \item Séverin Lemaignan, CHILI Lab, École Polytechnique Fédérale de
    Lausanne, \texttt{severin.lemaignan@epfl.ch} 
    \item Kshitij Sharma, CHILI Lab, École Polytechnique Fédérale de Lausanne,
    \texttt{kshitij.sharma@epfl.ch}
    \item Ashish Ranjan Jha, CHILI Lab, École Polytechnique Fédérale de
    Lausanne, \texttt{ashish.jha@epfl.ch}
    \item Pierre Dillenbourg, CHILI Lab, École Polytechnique Fédérale de
    Lausanne, \texttt{pierre.dillenbourg@epfl.ch}
\end{itemize}

\section{Title}

\begin{center}
{\sc Cognitive Context: Impact on the Perception of a Robot}
\end{center}

\section{Contribution type}

Research article

\section{Keywords}

human-robot interaction, anthropomorphism, cognitive context, eye-tracking, gaze patterns

\section{Relationship with prior papers}

\begin{itemize}
\item Lemaignan, S., Fink, J., \& Dillenbourg, P. (2014). The dynamics of
anthropomorphism in robotics. In \emph{Proceedings of the 2014 human-robot
interaction conference}: this paper discusses the dynamics of anthropomorphic
attributions, in a completely different setting and with a completely
different methodology than the one used in the present publication.
\item Fink, J. (2012). Anthropomorphism and human likeness in the design of
robots and human-robot interaction. In \emph{S. S. Ge, O. Khatib, J.-J.
Cabibihan, R. Simmons, \& M.-A. Williams (Eds.), Social robotics}: this paper
introduce theoretical background on the concept of anthropomorphism which is
re-used in the introduction of the present publication.
\end{itemize}

\section{Implication of authors}

\begin{itemize}
\item Séverin Lemaignan: experiment design
\item Kshitij Sharma: data analysis
\item Ashish Jha: data collection
\item Pierre Dillenbourg: project supervision
\end{itemize}

\section{JHRI Sections}

\textbf{Preferred section}: \emph{Behavioral and Social Science}

\textbf{Possible Editors}: Vanessa Evers, Wendy Ju

\section{Presentation at HRI}

We will gladly present this work during a future HRI Conference.

\end{document}
